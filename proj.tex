% !TEX root = proj.tex
\documentclass[9pt]{amsart} \usepackage[utf8]{inputenc}

\title{A Borcherds lift of the weak Jacobi form $\phi_{0,1}$,
generalized Borcherds-Kac-Moody superalgebras and the Igusa cusp form $\Delta_5$}

\author{Raeez Lorgat } \date{April 2 2020}

%\usepackage[backend=bibtex,style=verbose-trad2]{biblatex}
%\bibliography{proj}
%\bibliographystyle{ieeetr}
\usepackage[margin=0.8in]{geometry}
\usepackage{mathpazo}
\usepackage{graphicx} \usepackage{amsmath}
\usepackage{amssymb}
\usepackage{amsthm}
\usepackage{tikz-cd}

%\usepackage{natbib}
\newtheorem{theorem}{Theorem}
\newtheorem{lemma}{Lemma}
\newtheorem{remark}{Remark}

\newcommand{\Imag}{\mathrm{Im}}
\newcommand{\Real}{\mathrm{Re}}
\newcommand{\Proj}{\mathbb{P}}
\newcommand{\N}{\mathbb{N}}
\newcommand{\Q}{\mathbb{Q}}
\newcommand{\Z}{\mathbb{Z}} \newcommand{\C}{\mathbb{C}}
\newcommand{\R}{\mathbb{R}} \newcommand{\La}{\Lambda}
\newcommand{\HypPlan}{\Lambda^{(1,1)}}
\newcommand{\Sp}{\mathbf{Sp}}
\newcommand{\PSp}{\mathbf{PSp}}
\newcommand{\GL}{\mathbf{GL}}
\newcommand{\SL}{\mathbf{SL}}
\newcommand{\Orth}{\mathbf{O}}
\newcommand{\SO}{\mathbf{SO}}
\newcommand{\Hpl}{\mathcal{H}}
\newcommand{\IV}{\mathbf{IV}}
\newcommand{\fbasis}{(f_i)_{\{1,2,3,-2,-1\}}}
\newcommand{\Id}{\mathbf{I}}
\newcommand{\Cone}{\mathcal{C}}
\newcommand{\Poly}{\mathcal{P}}
\newcommand{\bkm}{\mathfrak{g}}
%\newcommand{\bkm}{\mathfrak{g}_{\begin{pmatrix}2 & -2 & -2 \\
%-2 & 2 & -2\\
%-2 & -2& 2\end{pmatrix}}}
\newcommand{\autcor}{\mathfrak{g}_{\Delta_5}}

\begin{document}

\maketitle

\begin{abstract} We investigate the relationship between the siegel
modular form known as the Igusa cusp form $\Delta_5$ of weight $5$ and a
certain weak Jacobi form $\phi_{0,1}$ of weight $0$ and index $1$. Following
Kac-Moody, Borcherds, Gritsenko and Nikulin et. al., we work out in
elementary detail the derivation of an infinite product formula for
$\Delta_5$ as an expansion at the cusp. In doing so, we give an elementary
construction of a pair of an infinite dimensional generalized kac-moody
superalgebra $\bkm$ along with its \textit{automorphic
correction} $\bkm\subset
\autcor$, both constructed from an underlying
real (in the case of $\bkm$) (respectively real and
imaginary in the case of $\autcor$) root datum.
Both of these sets of root datum are realized in the geometry of the
lattice $\La^{3,2}$ of signature $(3,2)$, its hyperbolic
sublattices $\La^{2,1}, \La^{2,1}_{II}$,the associated Weyl group(s) and
Weyl vector. The macdonald identity for the automorphic correction
algebra witnesses the infinite product formula for the cusp form
$\Delta_5$ via the Weyl-Kac character formula applied to the trivial one
dimensional representation $\C$. The weight 0 index 1 weak Jacobi form
$\phi_{0,1}$ is related to $\Delta_5$ as a Jacobi form counting the
super dimensions of weight spaces in the automorphic
correction $\autcor$.  \end{abstract}
% We conclude with a brief
% interpretation of the dimensions of the weight spaces of
% $\autcor$ as physical states in a string
% propagating in a background Calabi-Yau geometry given by $S \times E$
% where $S$ is a $K3$ surface and $E$ is an elliptic curve. This
% perspective motivates us to give a short list of formal generating
% functions of coherent ideal sheaves of algebraic Calabi-Yau three-folds,
% generalizing the structure found in the case of $S\times E$ and
% $\Delta_5$.

 % \section{Notation and Conventions}
 %
 % \begin{itemize}
 % \item $\Sp_4(\Z)$ definition and generators
 % \item $\La^{3,2}$
 % \item $\La^{1,1}$
 % \item $\fbasis$
 % \item $Z = (z1,z2, newline z2,z3)$
 % \item The $n\times n$ identity matrices given by $\Id_n$
 % \item $\O_{\R}$ has four components
 % \item $\O_{\R,+}$
 % \item type $\IV$ domains $\Hpl^{\IV}_+,\overline{\Hpl^{\IV}_+}$
 % \item
 % \end{itemize}

%\section{Overview}


% TODO writeup overview
% - infinite product formulas and distribution of zeros / poles
% - geometry of inclusion of cusp
% - expansion of weyl-kac character formula of kac-moody algebras "at
%   the cusp"
% - explicit cartan matrix and kac-moody algebras (recall that every kac-moody
%
% - how to discuss singular theta lifts, Harvey-Moore, automorphic
%   (threshold) corrections, borcherd's automorphic forms with
%   singularities on grassmannians, leading up to the gritsenko-maas
%   (additve) borcherds (multiplicative) lifts
%
% - how to treat the same, while keeping faithful to stringy
%   motivations/applications

\tableofcontents

\section{Preamble on the cusp form $\Delta_5$}

We freely use results from Freitag\cite{FREITAG:1} and Van der
Geer\cite{VDGEER:1}. Recall the ring of Siegel modular
forms $$\mathcal{SM}(\Sp_4(\Z)) = \C[E_4,E_6,\Delta_{10},\Delta_{12}]$$
generated by two eisenstein series $(E_4,E_6)$ of weights 4 and 6 and
the two siegel cusp forms $\Delta_{10},\Delta_{12}$ of weights 10 and 12
respectively. Note that $\Delta_{10}$ is the square of a cusp form
$\Delta_5$ of weight 5 with a non-trivial multiplier system $\nu_{\Delta_5}$.
$\Delta_5$ may be explicitly expressed as the product of all ten even theta
constants $$\Delta_5 = \displaystyle\prod_{\substack{(a,b)\in (\Z/2\Z)^2\\{}^t ab
\equiv 0 \mod 2}}\nu_{a,b}$$
where $$\nu_{ab}(z) =\displaystyle\sum_{l\in\Z^2} \exp(\pi i (z[l +
\frac{1}{2}a] + {}^tbl))$$ using $$z[x] = {}^tx z x.$$

Let $M = \begin{pmatrix} A & B\\
C & D\end{pmatrix} \in \Sp_4(\Z)$, then the explicit form of the
non-trivial multiplier system $$\nu_{\Delta_5}:
\Sp_4(\Z) \rightarrow \C$$ for $\Delta_5$
such that $|\nu_{\Delta_5}(g)| = 1$ for all $g \in \Sp_4(\Z)$
(found by Maass \cite{MAASS:1}) is given by
$$\nu_{\Delta_5}\begin{pmatrix}0 & \Id_2\\
-\Id_2 & 0\end{pmatrix} = 1,$$ $$\nu_{\Delta_5}\begin{pmatrix}\Id_2 & B\\
0 & \Id_2\end{pmatrix} = (-1)^{b_1 + b_2 + b_3},$$
$$\nu_{\Delta_5}\begin{pmatrix}{}^tA^{-1} & 0\\
0 & A\end{pmatrix} = (-1)^{(1 + a_1 + a_4)(1 + a_2 + a_3) + a_1a_4}$$
where $A = \begin{pmatrix}a_1 & a_2\\
a_3 & a_4\end{pmatrix} \in \GL_2(\Z)$ and $B = \begin{pmatrix}b_1 &
b_2\\
b_2 & b_3\end{pmatrix} \in M_{2\times2}(\Z).$



Using the expression in terms of even theta constants as well as the
explicit form of the multiplier system, we can show that
in terms of the matrix $Z=\begin{pmatrix}z_0 & z_1\\ z_2 &
z_3\end{pmatrix}$ that we can express the fourier expansion $$\Delta_5(Z) =
\displaystyle\sum_{\substack{n,l,m = 1 \textrm{mod} 2,\\4nm - l^2 > 0,\\ n,m > 0}} f(n,l,m)\exp(\pi
i(n z_1 + l z_2 + mz_3))$$ It is easy to show that $f(1,1,1) = 64$ as
well as that $64 | f(n,l,m)$ for all $n,l,m$.  Finally, we
will need the following identity of power series

$$1 + \frac{1}{64} \displaystyle\sum_{n\in \mathbb{N}} f(1+2t, 1, 1)q^t
= \displaystyle\prod_{k\in
\mathbb{N}}(1 - q^k)^9$$

\begin{proof}
Recall the fourier-Jacobi expension of $$\Delta_5(Z) =
\displaystyle\sum_{\substack{m>0,\\m = 1
\mod 2}} \phi_{5,m} (z_1,z_2)\exp (\pi i m z_3)$$
then the first Fourier-Jacobi coefficient is a Jacobi cusp form of index
$\frac{1}{2}$ and non-trivial character. We'll need the Jacobi
theta-series $$\nu_{11}(z_1,z_2) = \displaystyle\sum_{n \in \Z} (-1)^n \exp(\frac{\pi
i}{4} (2n + 1)^2 z_1 + \pi i(2n +1) z_2),$$ a variant of the Jacobi
triple-product formula yields a product expansion for $$\nu_{11} =
q^{\frac{1}{8}}r^{-\frac{1}{2}} \displaystyle\prod_{n \geq 1} (1 - q^{n-1} r ) (1 - q^n
r^{-1}) (1 -q^n)$$ where $q = \exp(2\pi i z_1)$ and $r = \exp(2\pi i
z_2)$, but
then $\psi_{5,\frac{1}{2}} = \eta(z_1)^9 \nu_{11}(z_1,z_2)$ is another
Jacobi cusp form of index $\frac{1}{2}$ and the same character, with
$$\eta(z_1) = \exp(\frac{\pi i z_1}{12}) \displaystyle\prod_{n\geq 1} (1 - \exp(2\pi i n
z_1);$$ the
squares of these Jacobi cusp forms are Jacobi cusp forms of weight $10$
and index $1$; up to a constant, there is only one of these, and it is
the first Fourier-Jacobi coefficient of $\Delta_{10} = \Delta_5^2$;
comparing fourier coefficients of the product expansion we obtain $$\frac{1}{64}
\phi_{5,1}(z_1,z_2) = \psi_{5,\frac{1}{2}}(z_1,z_2) =
-q^{\frac{1}{2}}r^{-\frac{1}{2}}\displaystyle\prod_{n\geq 1} (1 - q^{n-1}r )(1 -q^n
r^{-1})(1 - q^n)^{10}$$ the
desired identity as an application of the Jacobi triple-product identity
applied to the coefficient of $r^{\frac{1}{2}}$.
\end{proof}
Together these fundamental properties of the cusp form $\Delta_5$
will be used to construct a generalized kac moody lie superalgebra with
denominator identity in terms of $\Delta_5$.


\section{An isomorphism between
the symplectic group $\Sp_4(\Z)/\{\pm\Id_5\}$ and the orthogonal group
$\Orth(\La^{3,2})_+/{\pm\Id_5}$}

Proofs for results of this section can be found in \cite{KNAPP:1}. We
start with some isomorphisms relating relevant low rank symplectic and
orthogonal groups.


Consider the rank 4 free $\Z$-module $$\La^4 = \Z e_1 \oplus \Z e_2
\oplus \Z e_3 \oplus \Z e_4.$$

Any $\Z$-linear map $g: \La^4 \rightarrow \La^4$ induces a linear map
$\wedge^2 g : \La^4 \wedge \La^4 \rightarrow \La^4 \wedge \La^4.$ In
particular, we have an induced action of $\SL_4(\Z)$.

We have a (pfaffian) scalar product
$(,): \La^4 \wedge \La^4 \rightarrow \C$ defined by $u \wedge v =
(u,v)e_1 \wedge e_2 \wedge e_3 \wedge e_4 \in \wedge^4 \La^4$.  This is
an $\SL_4(\Z)$ invariant even unimodular integral symmetric bilinear
form of signature $(3,3)$.

Observe that the for $q = e_1\wedge e_3 + e_2 \wedge e_4 \in \La^4
\wedge \La^4$ we have $$-x \wedge y \wedge q = B_q(x,y) e_1 \wedge e_2
\wedge e_3 \wedge e_4$$ and hence the elements $q \in \La^4 \wedge
\La^4$ can be identified with integral skew-symmetric bilinear forms on
$\La^4$.


It follows that $$ \{ g : \La^4 \rightarrow \La^4 | (g \wedge g)(e_1
\wedge e_3 + e_2 \wedge e_4) = e_1 \wedge e_3 + e_2 \wedge e_4 \} \simeq
\Sp_4(\Z).$$

Hence the lattice $$\La^{3,2} = (e_1 \wedge e_3 + e_2 \wedge e_4)^{\perp}
\subset \La^4 \wedge \La^4$$ is isomorphic to $$\La^{3,2} \simeq \HypPlan \oplus \HypPlan \oplus [2]$$
with $[2]$ the one dimensional $\Z$ lattice with inner product given by
the matrix $(2)$ and $\HypPlan$ the standard integral hyperbolic plane i.e.\ a lattice with
quadratic form $\begin{pmatrix}0 & -1\\ -1 & 0\end{pmatrix}$

Next we fix a basis $\fbasis$ in $\La^{3,2}$ given by
$$ (f_1 = e_1 \wedge e_2,$$
$$ f_2 = e_2 \wedge e_3,$$
$$ f_3 = e_1 \wedge e_3 - e_2 \wedge e_4, $$
$$ f_{-2} = e_4 \wedge e_1, $$
$$ f_{-1} = e_4 \wedge e_3)$$

Now the real orthogonal group $\Orth_{\R}(\La^{3,2}) =
\Orth_{\R}(\La^{3,2}\otimes \R)$ acts on the domain $$\Hpl^{\IV} = \{ Z \in
\Proj(\La^{3,2} \otimes \C) | (Z,Z) = 0, (Z, \overline{Z}) < 0\} = \Hpl_+^{\IV}
\cup \overline{\Hpl^{\IV}_+}$$

where we have (in the basis of the $\fbasis$ given above
$$\Hpl_+^{\IV} = \{ Z = {}^t((z_2^2 - z_1 z_3), z_3, z_2, z_1, 1)\cdot z_0
\in \Hpl^{\IV} |\Imag(z_1) > 0 \},$$ is the classical homogenous domain
of type $\IV$. Note that the condition $(Z,\overline{Z})<0$ is equivalent to $y_1 y_3 -
y_2^2 > 0$ where the $y_i = \Imag(z_i)$.

Then it is easy to see that the domain $\Hpl_+^{\IV}$ coincides with the
Siegel upper half plane, $\Hpl_2$ after we identify points of
$\Hpl_+^{\IV}$ with symmetric matrices $\begin{pmatrix} z_1 & z_2\\ z_2 &
z_3\end{pmatrix}$.

The real orthogonal group $\Orth_{\R}(\La^{3,2})$ has four connected
components. We denote by $\Orth_{\R}(\La^{3,2})_+$ the subgroup of
$\Orth_{\R}(\La^{3,2})$ of index $2$ consisting of those elements which
leave $\Hpl_+^{\IV}$ invariant. The kernel of the action of
$\Orth_{\R}(\La^{3,2})_+$ on $\Hpl_+^{\IV}$ is given by $\pm \Id_5$. Since
$\La^{3,2}$ is odd-dimensional, the group $\Orth_{\R}(\La^{3,2})_+ =
\pm\Id_5\SO_{\R}(\La^{3,2})_+$ where $\SO_{\R}(\La^{3,2})_+$ is the
subgroup of elements with real spin-norm equal to $1$. Then we denote
$$\Orth(\La^{3,2})_+ = \Orth_{\R}(\La^{3,2})_+ \cap \Orth(\La^{3,2})$$
and $$\SO(\La^{3,2})_+ = \SO_{\R}(\La^{3,2})_+ \cap \Orth(\La^{3,2})$$

It is now an elementary exercise to realize concretely the images of the
generators of $\Sp_4(\Z)$ given for $M=\begin{pmatrix} m_1 & m_2\\
m_2 & m_3\end{pmatrix}$ such that $M \in M_{2\times2}(\Z)$,
$$\wedge^2(g_0) = \wedge^2(\begin{pmatrix}0 & \Id_2 \\ -\Id_2 &
0\end{pmatrix} = \begin{pmatrix}
0 & 0 & 0 & 0 & -1\\
0 & 0 & 0 & -1 & 0\\
0 & 0 & 1 & 0 & 0\\
0 & -1 & 0 & 0 & 0\\
-1 & 0 & 0 & 0 & 0
\end{pmatrix},$$
$$\wedge^2(g_{M \in M_{2\times2}(\Z)}) = \wedge^2(\begin{pmatrix}\Id_2 & M\\ 0 &
\Id_2\end{pmatrix} = \begin{pmatrix}1 & -m_1 & 2m_2 & -m_3 & m^2 - m_1 m_2\\
0 & 1 & 0 & 0 & m_3\\
0 & 0 & 1 & 0 & m_2\\
0 & 0 & 0 & 0 & m_1\\
0 & 0 & 0 & 0 & 1\end{pmatrix},$$
$$\wedge^2((g_A)_{A\in GL_2(\Z)}) = \wedge^2(\begin{pmatrix}{}^tA^{-1} &
0\\0 & A\end{pmatrix} = \text{det}(A)\begin{pmatrix}1 & 0 & 0 & 0 & 0\\
0 & a_1^2 & -2a_1a_2 & a_2^2 & 0\\
0 & -a_1a_3 & a_1a_4 + a_2a_3 & -a_2a_4 & 0\\
0 & a_3^2 & -2a_3a_4 & a_4^2 & 0\\
0 & 0 & 0 & 0 & 1\end{pmatrix}$$

Hence we have

\begin{lemma}
The correspondence $\wedge^2$ defines an isomorphism $$\wedge^2 :
\Sp_4(\Z)/\{\pm \Id_5\} \rightarrow \SO_+(\La^{3,2}) \simeq
\Orth(\La^{3,2})_+/\{\pm\Id_5\}$$ yielding a commutative square

  \[ \begin{tikzcd}
\Hpl_2 \arrow{r}{} \arrow[swap]{d}{} & \Hpl_2 \arrow{d}{} \\%
\Hpl_+^{\IV} \arrow{r}{g \wedge g}& \Hpl_+^{\IV}
\end{tikzcd}
\]


involving isomorphisms $\Hpl_2 \rightarrow \Hpl_+^{\IV}$ and arbitrary
$g \in \Sp_4(\Z).$

\end{lemma}

\section{$\Delta_5$ and the lattice $\La^{3,2}$}

Consider some fixed primitive hyperbolic sublattice
$$\La^{2,1} = \HypPlan \oplus [2] \simeq \Z f_2 \oplus \Z f_3 \oplus \Z
f_{-2} \subset \La^{3,2}$$

Extending automorphisms $\phi \in \Orth(\La^{2,1})$ to be the identity
on $(\La^{2,1})^{\perp}$ yields an embedding $\Orth(\La^{2,1})
\rightarrow \Orth(\La^{3,2})$, hence we can investigate the automorphy
of $\Delta_5$ with respect to the subgroup $\Orth(\La^{2,1})$.

%TODO definition of primitive element
% TODO expand on: every primitive element (or any other integral symmetric
% bilinear form) with $(\alpha,\alpha) > 0 and (\alpha,\alpha)|2(\La^{2,1},\alpha)$

Recall for example from Kac \cite{KAC:1} that to every primitive element $(\alpha \in
\La^{2,1})$ satisfying $(\alpha,\alpha) > 0$ and $(\alpha, \alpha) |
2(\La^{2,1}, \alpha)$ defines a reflection $$s_{\alpha}: x \mapsto
2\frac{(x,\alpha)}{(\alpha, \alpha)}\alpha$$ for all $x \in \La^{2,1}$. Then
$s_{\alpha}(\alpha) = -\alpha$ and $s_{\alpha}|_{\alpha^{\perp}}$ is the
identity. Hence for all $\alpha \in \La^{2,1}$ satisfying
$(\alpha,\alpha) = 2$ we get a reflection $$s_{\alpha}: x \mapsto
(x,\alpha)\alpha.$$

Now observe
\begin{lemma}
Consider $\alpha$ an element with square $2$, then if $\alpha \in
\{\delta_1 = 2f_2 -f_3, \delta_2 = 2f_{-2} - f_3, \delta_3 = f_3\}$ we
have
$$\Delta_5(s_{\alpha}Z) = - \Delta_5(Z)$$
while if $\alpha \in \{f_{-2} - f_2, f_2 - f_3, f_2 + f_3\}$ then

$$\Delta_5(s_{\alpha}Z) = \Delta_5(Z)$$
\end{lemma}
\begin{proof}
Denoting by $\overline{U} = \wedge^2(\begin{pmatrix}{}^tU^{-1} & 0\\ 0 & U\end{pmatrix})$ with $U \in \GL_2(\Z)$ then we have
$$s_{f_{-2} - f_2} = -\overline{\begin{pmatrix}0 & 1\\ 1 &0\end{pmatrix}}, s_{f_3} = - \overline{\begin{pmatrix}1& 0\\  0 &-1\end{pmatrix}}, s_{f_2 - f_3} =
-\overline{\begin{pmatrix}1 & -1\\ 0 & -1\end{pmatrix}}.$$ Then the result follows from Maas' explicit
formula for the multiplier of $\Delta_5.$
\end{proof}

Now observe that since $\La^{2,1}$ is hyperbolic of signature $(2,1)$
then $\La^{2,1}$ defines a cone $\Cone(\La^{2,1}) = \{ x \in \La^{2,1}
\otimes \R | (x,x) < 0\}$. Then $\Cone(\La^{2,1})$ is a union of two
half-cones; We select one of these cones $\Cone(\La^{2,1})_+$ by the
constraint that the complexified cone $$\Omega(\Cone(\La^{2,1})_+) = \La^{2,1} \otimes \R
+i\Cone(\La^{2,1})_+ \subset \Hpl^{\IV}_+$$ hence for $$z = z_3 f_2 + z_2
f_3 + z_1 f_{-2} \in \Omega(\Cone(\La^{2,1}))$$ then the corresponding
point $$Z = {}^t ((z_2^2 - z_1z_3), z_3, z_2, z_1, 1) \cdot z_0 \in
\Hpl^{\IV}_+.$$ Denote by $\Orth(\La^{2,1})_+$ the subgroup of
$\Orth(\La^{2,1})$ of index 2 fixing the half-cone
$\Cone(\Lambda^{2,1})_+$. Then since $\La^{2,1}$ is a hyperbolic
lattice, the group $\Orth(\La^{2,1})_+$ is discrete in the corresponding
hyerbolic space $$\Cone(\La^{2,1})_+ / \R_{>0}$$ where we take the
qotient by the strictly positive real numbers' scaling action.
Hyperbolicity also implies that the hyperbolic space
$\Cone(\La^{2,1})_+ / \R_{>0}$ has a fundamental domain of finite
volume. It follows that any reflection $s_{\alpha}$ with $\alpha \in
\La^{2,1}$ satisfying $(\alpha,\alpha) > 0$ is a reflection in the
hyperplane $$\Hpl_{\alpha} = \{ \R_{>0}x \in \Cone(\La^{2,1})_+/\R_{>0} |
(x,\alpha) = 0\}.$$ Hence this reflection maps the half-space
$$\Hpl_{\alpha,+} = \{\R_{>0} x \in \Cone(\La^{2,1})_+/\R_{>0} |
(x,\alpha) \leq  0\}$$ to the opposite half-space $\Hpl_{-\alpha}$ which
are both bounded by the hyperplane $\Hpl_{\alpha}$. We call $\alpha$
\textit{orthogonal} to both $\Hpl_{\alpha}$ and $\Hpl_{\alpha,+}$. Taken
together, all the reflections of $\Lambda^{2,1}$ generate a reflection
subgroup $$W(\La^{2,1}) \subset \Orth(\La^{2,1})_+.$$


The lattice $\La^{2,1}$ is special, in particular because the
automorphism group is known explicitly. Here we list the facts we will
need from \cite{NIKULIN:1}
The group $\Orth(\La^{2,1})_+ = W^{(2)}(\La^{2,1})$ where the index
$(2)$ indicates the subgroup generated by reflections in all elements of
$=\La^{2,1}$ with square $2$. Analagously, we define
$\Delta^{(k)}(\Lambda^{2,1})$ the set of all primitive elements
$\delta \in \La^{2,1}$ with $(\delta,\delta) = k$ which define
reflections $s_{\delta}$ of $\La^{2,1}$ and similarly $W^{(k)}$ denotes
the reflection group generated by all these reflections $s_{\delta}$.
Thus, we can reformulate what we have seen so far in this language, by
stating that $\Orth(\La^{2,1})_+$ is generated by reflections in
$\Delta^{(2)}(\La^{2,1})$ and any element of this set $\delta$
corresponds to a reflection $s_{\delta}$ of one of two types:
\begin{itemize}
\item Type I: $(\delta, \La^{2,1}) = \Z$
\item Type II: $(\delta, \La^{2,1}) = 2\Z$
\end{itemize}

We introduce sublattices $\La^{2,1}_I$ and $\La^{2,1}_{II}$ which are
generated by elements $\delta_I$ or $\delta_{II}$ of type $I$ and $II$
respectively. Then we have

$$\La^{2,1}_I = \{m f_2 + l f_3 + n f_{-2} \in \La^{2,1} | m + l + n = 0 \mod 2\}$$

and

$$\La^{2,1}_{II} = \{m f_2 + l f_3 + n f_{-2} \in \La^{2,1} | m = n =  0 \mod 2\}$$

and an element $\delta \in \Delta^{(2)}(\La^{2,1})$ has type $I$
(respectively type $II$) if and only if $\delta \in \La^{2,1}_I$
(respectively $\delta \in \La^{2,1}_{II}$). It follows that the
subgroups of $\Orth(\La^{2,1})_+$ generated by all reflections of type
$I$ (respectively type $II$) are given by $W^{(2)}(\La^{2,1}_I)$
(respectively $W^{(2)}(\La^{2,1}_{II})$.) All lattices $\La^{2,1},
 \La^{2,1}_I$ and $\La^{2,1}_{II}$ are $W^{(2)}(\La^{2,1})$ invariant and
 both subgroups of reflections of type $I$ and $II$ are normal in
 $W^{(2)}(\La^{2,1})$. The index of $W^{(2)}(\La^{2,1}_I)$ as a subgroup
 is 2, while the index of $W^{(2)}(\La^{2,1}_{II})$ is $6$. We have
 fundamental polyhedra $\Poly, \Poly_I$ and $\Poly_{II}$ for each
 respective reflection group; denoting by $\Poly' \in \{\Poly, \Poly_I,
 \Poly_{II}\}$ we can express these polyhedra explictly in each case by
 the intersection of all hyperplanes $$\displaystyle\bigcap_{\delta \in
 \Poly'_{prim}} \Hpl_{\delta,+}$$ determined by minimal sets of
 primitive elements of positive square (in this case, square $2$) in
 $\La^{2,1}$. These sets are given explicitly by $\Poly'_{prim}$ of
 (primitive) orthogonal vectors to the given polyhedra by

 $$\Poly_{prim} = \{f_2 - f_3, f_{-2} - f_2, f_3\}$$
 $$\Poly_{I,prim} = \{f_2 -f_3, f_{-2} - f_2, f_2 + f_3\}$$
 $$\Poly_{II,prim} = \{\delta_1, \delta_2, \delta_3\}$$
It follows that the three types of reflection groups $W^{(2)}(\La^{2,1}),
W^{(2)}(\La^{2,1}_I)$ and $W^{(2)}(\La^{2,1}_{II})$ are generated by
reflections in faces of each of the fundamental polyhedra respectively. We denote
by $$Aut(\Poly') = \{g \in \Orth(\La^{2,1})_+ | g \Poly' = \Poly' \}$$ the
group of symmetries of each fundamental polyhedron $\Poly'$, then the group
$Aut(\Poly_{prim})$ is trivial, while the group $Aut(\Poly_{I,prim})$ has order
$2$ and is generated by $s_{f_3}$, while the group $Aut(\Poly_{II,prim})
\simeq S_3$ and is generated by $s_{f_2 -f_3,}, s_{f_{-2} - f_2}$. So we
can realize $$\Orth(\La^{2,1})_+ \simeq W^{(2)}(\La^{2,1}) \simeq
W^{(2)}(\La^{2,1}_I) \rtimes Aut(\Poly_{I,prim}) \simeq
W^{(2)}(\La^{2,1}_{II}) \rtimes Aut(\Poly_{II,prim})$$

To summarize, the automorphy of $\Delta_5$ with respect to subgroups of
$\Orth(\La^{2,1})$ can be expressed as

\begin{lemma}
$\Delta_5$ is either invariant or anti-invariant with respect to elements
of the group $\Orth(\La^{2,1})_+$. By our explicit classification above, we can
distinguish two cases:
\begin{itemize}
  \item when $w \in W^{(2)}(\La^{2,1}_I)$ and $a \in Aut(\Poly_{I,prim})$ we
  have
  $$\Delta_5(w \cdot a Z) = \text{det}(a) \Delta_5(Z)$$
  \item when $w \in W^{(2)}(\La^{2,1}_{II})$ and $a \in Aut(\Poly_{II,prim})$ we
  have
  $$\Delta_5(w \cdot a Z) = \text{det}(w) \Delta_5(Z)$$
\end{itemize}

\end{lemma}
%\begin{proof}
%\end{proof}

Next let us investigate the cone $\Delta(\La^{2,1}_{II})_+ = \R_{\geq 0} \delta_1 + \R_{\geq 0} \delta_2
+ \R_{\geq 0} \delta_3$ along with its dual cone
$$\Delta(\La^{2,1}_{II})_+^* = \{ x \in \La^{2,1} \otimes \R | (x,
\delta_i) \leq 0 \}.$$ Since $\Poly_{II} \subset \Cone(\La^{2,1})_+ /
\R_{>0}$ has finite volume in the hyperbolic space and since the cone
$\overline{\Cone(\La^{2,1})_+} = \overline{\Cone(\La^{2,1})_+^*}$ is
self-dual, the above is equivalent to the sequence of embeddings of
cones
$$\Delta(\La^{2,1}_{II})_+^* \subset
\overline{\Cone(\La^{2,1})_+} \subset  \Delta(\La^{2,1}_{II})_+$$
We stress this property is key to our construction, and is actually equivalent to
finite volume of $\Poly_{II}$.

Another important property of the group $W^{(2)}(\Poly_{II})$ is the
existence of a \textit{Lattice Weyl vector}. This is an element $\rho
\in \Poly_{II} \otimes \Q$ satisfying $$(\rho, \delta_i) =
-\frac{(\delta_i,\delta_i)}{2} = -1$$ for any $\delta_i \in \Poly_{II}.$
But then by the Gram matrix of the $\delta_i$

$$(\delta_i,\delta_j) = \begin{pmatrix}2 & -2 & -2\\-2 & 2 & -2\\-2 & -2
& 2\end{pmatrix}$$
we have $$\rho = \frac{1}{2}\delta_1 + \frac{1}{2}\delta_2 +
\frac{1}{2}\delta_3 = f_2 - \frac{1}{2}f_3 + f_{-2}$$

Identifying $\La^{2,1}\otimes \Q \simeq \La^{2,1}_{II} \otimes \Q$,
then it is clear that $\rho \in \Delta(\La^{2,1}_{II})_+^*$ hence
by the sequence of embeddings of cones above we have $\rho \in
\Cone(\La^{2,1})_+ = \Cone(\La^{2,1}_{II})_+$.

Now we use the reflection group $W^{(2)}(\La^{2,1}_{II})$ to study the
fourier coefficients of $\Delta_5$. Again fix  $z = z_1 f_{-2} + z_2 f_3 +
z_3 f_2 \in \La^{2,1} \otimes \R + i\Cone(\La^{2,1})_+ = \La^{2,1}_{II}
\otimes \R + i\Cone(\La^{2,1}_{II})_+$ as above. Then the lattice
$$(\La^{2,1}_{II})^* = \Z \frac{1}{2} f_2 + \Z \frac{1}{2} f_3 + \Z
\frac{1}{2} f_{-2} = \frac{1}{2}\La^{2,1}$$

Thus for $n,l,m \in \Z$ we have $n \equiv m \equiv l \equiv 1 \mod 2$,
$n,m > 0$ and
$4nm - l^2 > 0$, and hence we can express $$\frac{1}{64}f(n,l,m)\exp(\pi i
(n z_1 + lz_2 + m z_3)) $$ $$= \frac{1}{64} f(n,l,m) \exp(- \pi i (nf_2 - l f_3
\frac{1}{2} + mf_{-2}, z))$$
$$= m(a) \exp (-\pi i(\rho +a, z)),$$
where $$a = (n-1) f_2 - (l-1)\frac{1}{2} f_3 + (m-1) f_2 \in (\La^{2,1})^*
= \frac{1}{2} \La^{2,1}_{II}$$ and $$m(a) = -\frac{1}{64} f(n,l,m).$$ By
the properties at the end of the pre-amble on the cusp form $\Delta_5$, we
see $\rho + a \in \Cone(\La^{2,1})_+$, $m(a) \in \Z$ and $m(0) = -1$.

Expressing the type $II$ case in the third lemma above in terms of the
lattice weyl vector $\rho$, the reflection group
$W^{(2)}(\La^{2,1}_{II})$ and the $m(a)$, we see

$$\frac{1}{64} \Delta_5(Z) = \displaystyle\sum_{w \in W^{(2)}(\La^{2,1}_{II})}
\text{det}(w) (
\displaystyle\sum_{\rho +a \in (\La^{2,1}_{II})^* \cap \La^{2,1}_{II}} m(a) \exp(-\pi
i(w(\rho +a), z) ))$$

since for this sum we have $\rho +a \in (\La^{2,1}_{II})^* \cap
\La^{2,1}_{II}$ we see $(\rho +a, \delta_i) \leq 0$ for all $i$. If
$(\rho + a, \delta_i) = 0$ then the corresponding fourier coefficient
$m(a) = 0$ since $\Delta_5$ is anti-invariant with respect to
$s_{\delta_i}$. It follows that $(\rho +a, \delta_i)$ is integral and
$(\rho +a, \delta_i) < 0$ for all $i$. By the construction of $\rho$ we
have $(a, \delta_i) \leq 0$ and it follows then that $a \in
\Delta(\La^{2,1}_{II})^*_+$ but then by the sequence of cone embeddings
above we have $$a \in \R_{\geq 0} \Cone(\La^{2,1}_{II})_+^* \cap
\overline{\Cone(\La^{2,1})_+} = \R_{\geq 0} \Poly_{II}.$$ It follows
that $a \in \R_{>0}\Poly_{II}$ if $a \neq 0$. If $a = 0$ we have $m(a) =
-1$, so by the congruences $m\equiv n \equiv l \equiv 1 \mod 2$ we have $a \in 2
(\La^{2,1})^* = \La^{2,1}_{II}.$


Now, putting this all together, and changing the range of summation to
$a \in \La^{2,1}_{II} \cap \R_{> 0}\Poly_{II}$, we find
\begin{lemma}
$$\displaystyle\sum_{w \in W^{(2)}(\La^{2,1})} \text{det}(w)( \exp(-\pi
i(w(\rho),z))- \sum_{a \in \La^{2,1}_{II}\cap\R_{>0}\Poly_{II}}
m(a)\exp(-\pi i(w(\rho+a),z)))$$
\end{lemma}

Now we investigate the behaviour of a certain triple of primitive
elements at \textit{infinity}. Their relevance will become clear
shortly. Consider the primitive elements $a_0 \in \La^{2,1}_{II} \cap
\R_{>0} \Poly_{II}$ with $(a_0,a_0) = 0$ corresponding to the three
vertices at infinity of the hyperbolic plane; by the third lemma above, the group
$A(\Poly_{II,prim})$ is transitive on these three vertices and the
corresponding primitive elements are given explicitly by $2f_2, 2f_{-2},
2f_2 - 2f_3 + 2 f_{-2}$); furthermore, by the third lemma above, the
group $A(\Poly_{II,prim})$ preserves the fourier expansion just
obtained. Thus given say $a_0 = 2 f_2$ one of the three primitive
elements of $\La^{2,1}_{II} \cap \R_{>0}\Poly_{II}$ we have the identity
of formal power series

$$1 + \frac{1}{64} \sum_{t \in \N} f(1 + 2t, 1, 1)q^t = \displaystyle\prod_{k\in \N} (1
- q^k)^9$$

is equivalent to an equality

$$ 1 - \sum_{t \in \N} m(t a_0) q^t = \displaystyle\prod_{t\in \N}
(1 - q^t)^{\tau(t a_0) = 9 }$$

where $\tau(a) = 9$ for any  $a \in \La^{2,1}_{II} \cap \R_{>0}
\Poly_{II}$ with $(a,a) = 0$. Then transitivity
of the group $S_3$ means it is true for all three primitive elements.


The just-derived expression for the fourier transform of $\frac{1}{64}
\Delta_5$, as well as the generating series and product in $q^t$ arising
from multiples of the primitive elements at infinity sharply recalls to
mind the Weyl-Kac denominator formula in the theory of generalized
kac-moody algebras. Indeed, the fundamental polyhedron $\Poly_{II}$
along with the set of orthogonal vectors $\Poly_{II,prim} =
\{\delta_1,\delta_2,\delta_3\}$ can be considered as a (real) root datum
for
the the Gram matrix

$$(\delta_i,\delta_j) = \begin{pmatrix}2 & -2 & -2\\-2 & 2 & -2\\-2 & -2 & 2\end{pmatrix}$$

\section{The generalized Kac-Moody algebra $\bkm$ and its automorphic
correction $\autcor$}

The Gram matrix of the elements $\Poly_{II,prim}$ is integral, has only $2$
on the diagonal and only non-positive integers off the diagonal, and
hence it is symmetric generalized Cartan matrix. The theory of Kac-Moody
algebras associates to this matrix an infinite dimensional lie algebra we will denote
$\bkm$. However, the structure of the lattices
$\La^{3,2}, \La^{2,1}$ and $\La^{2,1}_{II}$ as well as the fourier
coefficients of $\Delta_5$ strongly suggests the
presence of a strictly larger \textit{corrected} lie algebra.
Constructions of Borcherds along this line inform the construction of
this \textit{automorphic correction} $\autcor$.

Using the coefficients $m(a)$ and $\tau(a)$ of the last section, we
introduce the following sets of \textit{simple imaginary roots} for $a \in
\La^{2,1}_{II} \cap \R_{>0} \Poly_{II}$:
$$\Delta_{\overline{0}}^{im} = \{\tau(a)a | (a,a) = 0,\tau(a)>0\}$$
where $ka$ for $k\in \N$ means that we repeat $a$ exactly $k$ times and
$$\Delta_{\overline{1}}^{im} = \{ m(a)a | (a,a) < 0 , m(a)< 0\}$$
where $ka$ for $-k \in \N$ means we repeat the element $a$ exactly $-k$
times. The negative sign here corresponds to all of the $-k$ elements
being \textit{odd}, in the sense of superalgebras, hence the elements
$\Delta_{\overline{0}}^{im}$ and $\Delta_{\overline{1}}^{im}$ are the
\textit{even and odd imaginary simple roots} of a new, strictly larger algebra
$\mathfrak{g}_{\Delta_5}$.
Then we denote the imaginary simple roots by $$\Delta^{im} = \Delta_{\overline{0}}^{im} \cup \Delta_{\overline{1}}^{im}$$
Following from the kac-moody construction of $\bkm$, we
have $$\Delta_{\overline{0}}^{re} = \Delta^{re} = \Poly_{II,prim} = \Z\delta_1 \oplus \Z\delta_2 \oplus \Z\delta_3$$ which are
the \textit{real even simple roots}. Hence
$\autcor$ is a superalgebra \textit{without real odd
roots}. By construction, elements of $\Delta^{re}$ correspond to
elements of $\La^{2,1}_{II} \subset \La^{2,1}_{II}\otimes \R$.
Furthermore we observe $(\alpha,\alpha) > 0$ if $\alpha \in \Delta^{re}$
and $(\alpha,\alpha) \leq 0$ if $\alpha \in \Delta^{im}$ and
$(\alpha,\alpha') \leq 0$ for all distinct $\alpha, \alpha' \in \Delta$ if
$(\alpha,\alpha) > 0$ we also have
$$2\frac{(\alpha,\alpha')}{(\alpha,\alpha)} \in \Z$$ which is valid
because here $(\alpha,\alpha) = 2$ and $\alpha' \in \La^{2,1}_{II}$
where $\La^{2,1}_{II} = \{\delta_1,\delta_2,\delta_3\} = \Delta^{re}$.
Then our generalized kac-moody lie superalgebra
$\autcor$ is generated by $h_{\alpha},
e_{\alpha}, f_{\alpha}$ with $\alpha \in \Delta$.


Then the map $\alpha \mapsto h_{\alpha}$ gives an embedding of
$\La^{2,1}_{II}\otimes\R$ into $\autcor$ as an
even abelian subalgebra. We have the relations

\begin{itemize}
\item $[h_{\alpha}, e_{\alpha'}] = (\alpha, \alpha')e_{\alpha'}$ and
$[h_{\alpha}, f_{\alpha'}] = -(\alpha, \alpha')f_{\alpha'}$
\item $[e_{\alpha}, f_{\alpha'}] = h_{\alpha}$ if $\alpha = \alpha'$ and
is $0$ otherwise
\item $(\textrm{ad} e_{\alpha})^{1- 2\frac{(\alpha,\alpha')}{\alpha,\alpha)}}e_{\alpha'} = (\textrm{ad} f_{\alpha})^{1- 2\frac{(\alpha,\alpha')}{\alpha,\alpha)}}f_{\alpha'} = 0$
if $\alpha \in \Delta^{re}$
\item if $(\alpha,\alpha') = 0$ then $[e_{\alpha}, e_{\alpha'}] = [f_{\alpha},f_{\alpha'}] = 0$
\end{itemize}

The superalgebra $\autcor$ is graded by
$\La^{2,1}_{II}$. Let $$\widetilde{\Cone(\La^{2,1}_{II})_+} =
\sum_{\alpha \in \Delta} \Z_+\alpha \subset \La^{2,1}_{II}$$ be the
integral cone generated by all simple roots, which happens in this
instance to coincide with the integral cone of all simple real roots
(recall the sequence of embeddings of cones in the previous section).
Now we have the triangular decomposition $$\autcor = (\oplus_{\alpha \in \widetilde{\Cone(\La^{2,1}_{II})_+}} \mathfrak{g}_{\alpha}) \oplus
(\La^{2,1}_{II} \otimes \R) \oplus (\oplus_{\alpha \in
-\widetilde{\Cone(\La^{2,1}_{II})_+}} \mathfrak{g}_{\alpha})$$
 then $e_{\alpha}$ and $f_{\alpha}$ have degree
$\alpha$ and $-\alpha$ respectively, $\mathfrak{g}_0 = \La^{2,1}_{II}
\otimes \R$ and we call all elements $\alpha \in
\pm\widetilde{\Cone(\La^{2,1}_{II})_+}$ \textit{roots} if
$\mathfrak{g}_{\alpha}$ is nonzero. We define \textit{positive}
(respectively \textit{negative}) roots by $\Delta_{\pm} = \Delta \cap
\pm\widetilde{\Cone(\La^{2,1}_{II})_+}.$ For every root $\alpha$ we
denote $\textbf{mult}_{\overline{0}}\alpha =
\textrm{dim}\mathfrak{g}_{\alpha,\overline{0}}$ and
$\textbf{mult}_{\overline{1}}\alpha =
-\textrm{dim}\mathfrak{g}_{\alpha,\overline{1}}$ then
$$\textbf{mult}\alpha = \textbf{mult}_{\overline{0}}\alpha +
\textbf{mult}_{\overline{1}}\alpha = \textrm{dim}
\mathfrak{g}_{\alpha,\overline{0}} -
\textrm{dim}\mathfrak{g}_{\alpha,\overline{1}}.$$
Finally, we arrive at the denominator identity, which is the
Weyl-Kac-Borcherds character formula applied to the complex
one-dimensional representation of our generalized kac-moody
superalgebra; it reads $$\Phi = \sum_{w \in W^{(2)}(\La^{2,1}_{II})} \text{det}(w) (\exp(-2\pi i(w(\rho), z)) - \sum_{a\in \La^{2,1}_{II} \cap \R_{>0} \Poly_{II}} m(a) \exp(-2\pi i(w(\rho + a), z)) ) $$ $$= \exp(-2\pi i(\rho,z)) \displaystyle\prod_{\alpha \in \Delta_+} (1 - \exp(-2\pi i
(\alpha,z)))^{\textbf{mult}\alpha}$$ valid for $z \in
\Omega(\La^{2,1}_{II}) = \La^{2,1}_{II} \otimes \R + i
\Cone(\La^{2,1}_{II})_+$ and the function $\Phi$ is called the
denominator function.  Thus applying the results of the last section we
arrive at the infinite product expansion for the form $\Delta_5$:

\begin{theorem}
  $$\frac{1}{64}\Delta_5(2Z) = \Phi(z)$$
  and hence the denominator of the \textit{corrected} generalized
  kac-moody lie superalgebra $\autcor$ is the
  siegel modular form of genus $2$ and weight $5$.
\end{theorem}

The denominator function $\Phi$ is well-defined on the complexified cone
$\Omega(\Cone(\La^{2,1}_{II})_+)$ which admits embeddings as a cusp into
the type $\IV$ domain. The embedding is not canonical, as it is defined
up to changing $z\mapsto tz$ for $t\in \N$.
%accounting for the coefficient $2$ in the theorem.

\section{Super dimensions of root spaces and the weight 0 index 1 weak Jacobi
form $\phi_{0,1}$}

The last section demonstrates that there is a product formula for
$\Delta_5$:
$$\Phi = \sum_{w \in W^{(2)}(\La^{2,1}_{II})} \text{det}(w)
(\exp(-2\pi i(w(\rho), z)) - \sum_{a\in \La^{2,1}_{II} \cap \R_{>0}
\Poly_{II}} m(a) \exp(-2\pi i(w(\rho + a), z)) ) $$ $$= \exp(-2\pi i(\rho,z))
\displaystyle\prod_{\alpha \in \Delta_+} (1 - \exp(-2\pi i
(\alpha,z)))^{\textbf{mult}\alpha}$$

we will now combine automorphy of $\Delta_5$ with knowledge of certain
related Jacobi forms to determine the integers $\textbf{mult}\alpha$ for
roots $\alpha$.
Throughout we freely reference Eichler-Zagier\cite{EZ:1}.
Consider the Fourier-Jacobi expansion of $\Delta_{12}$ and denote by
$\phi_{12,1}$ the Jacobi cusp form of weight $12$ appearing as its first
fourier-Jacobi coefficient. Its fourier coefficients may be calculated
in terms of explicitly given Eisenstein series for the groups $\SL_2$
and the Jacobi group. The form has integral and coprime coefficients
$$\phi_{12,1}(z_1,z_2) = (r^{-1} + 10 + r)q + (10r^{-2} -88r^{-1} -132
-88r +10r^{2})q^2 + \ldots$$ we introduce another function with integral
coefficients $$\displaystyle\phi_{0,1} (z_1,z_2) =
\frac{\phi_{12,1}}{\delta_{12}} = \displaystyle\sum_{\substack{n\geq 0\\l\in \Z}}
f(n,l)\exp(2\pi i(nz_1 + lz_2)$$ where $$\delta_{12}(z_1) = q\displaystyle\prod_{n\geq
1}(1 - q^n)^{24}$$ is the $\SL_2(\Z)$ cusp form of weight $12$. Then
$\phi_{0,1}$ is a weak Jacobi form of weight $0$ and index $1$. It
satisfies the same functional equations as holomorphic Jacobi forms and
has nonzero coefficients only with indices $(n,l) \in \Z^2$ such that
$n\geq 0$ (as $\phi_{12,1}$ is a cusp form) and $4n - l^2 \geq -1$. The
weight is even, hence $f(n,l) = f(n,-l)$ and $f(n,l)$ only depends on
$4n-l^2$. Explicitly $$\phi_{0,1} (z_1,z_2) = (r^{-1} +10 + r) +
q(10r^{-2} - 64r^{-1} + 108  -64r + 10r^2) + \ldots$$
then we have the product expansion

\begin{theorem}
$$\frac{1}{64} \Delta_5 = \exp(\pi i(z_1 + z_2 + z_3)) \displaystyle\prod_{n,l,m \in \Z,
(n,l,m) >0} (1 - \exp(2\pi i(nz_1 + lz_2 + mz_3)))^{f(nm,l)}$$ where the
condition $(n,l,m) > 0$ means the product is taken over the set of
positive roots $\Delta_+$
\end{theorem}

\begin{remark}
Explicitly, the condition $(n,l,m) > 0$ means that $n\geq 0,m\geq 0$ and
$l$ is aritrary integral if $n>0$ or $m > 0$ and $l<0$ if $n = m = 0$.
\end{remark}

\begin{proof}
An analysis of the fourier coefficients of the form $\phi_{12,1}$ as
well as its expression as a linear combination of standard Jacobi theta
functions allows one to prove that the fourier coefficients of
$\phi_{0,1}$ have the asymptotic behavior of $$f(n,l) =O(\exp(\sqrt{4n
-l^2}).$$ This estimate together with the methodology of Kac\cite{KAC:1}
allows us to prove that the product in the theorem converges on any
neighborhood of the zero dimensional cusp of $\Sp_4(\Z)$.

We express the product in the theorem as $$\exp(\pi i(z_1 + z_2 + z_3))
\displaystyle\prod_{n>0,l\in\Z \textrm{or} n=0, l<0}(1 -\exp(2\pi i(nz_1 +
lz_2))^{f(0,l)}$$ $$ \times \displaystyle\prod_{n\geq0,m>0,l\in\Z}(1 - \exp(2\pi i(n z_1 + l
z_2 + m z_3)))^{f(nm,l)}$$ where we have $f(0,0) = 10, f(0,-1) =1,
f(0,l) = 0$ if $l < -1$.


Consider next the minus embedding of the usual hecke operators
$T_{-}(m)$ for $\GL_2(\Z)$ then for each Jacobi form $\phi$ of weight
$k$ and index $t \in \Q$ we have a function $(\phi |_k T_{-}(m))$ which
is a Jacobi form of index $mt$. One can proceed to show the factors of
the above product admits an expression of its logarithm as
$$\textrm{log}(\displaystyle\prod_{n\geq 0,m>0,l\in\Z} (\cdots) = - \sum_{m\geq 1} m^2
(\phi_{0,1}\exp^{2\pi i t z_3}|_0 T_{-}(m)))$$ The expansion shows that
this particular factor is invariant with respect to the maximal
parabolic subgroup $\Gamma_{\infty}$ of $\Sp_4(\Z)$.

  Now observe that the other factor in the product is equal to
  $\psi_{5,\frac{1}{2}}\exp(2\pi i t z_3)$. Together these facts imply
  the product transforms like a modular form for the Jacobi group
  $\Gamma_{\infty} / \pm\Id_4$ with a certain character $\nu_{\infty} :
  \Gamma_{\infty} / \pm\Id_4 \rightarrow \pm 1$. Testing against the
  matrix $I = \begin{pmatrix}\begin{matrix}0 & 1\\ 1 & 0\end{matrix} & 0\\ 0 &
  \begin{matrix}0 & 1\\1 & 0\end{matrix}\end{pmatrix}$

  yields antiinvariance  of the product, while the multiplier system
  returns $1$. It follows that since the Jacobi group and $I$ together
  generate $\PSp_4(\Z)$ we get that the product is a Siegel modular form
  of weight $5$ with the same multiplier system as $\Delta_5$. Comparing
  the first fourier coefficient finishes the proof.


\end{proof}


%\printbibliography
\bibliography{proj}
\bibliographystyle{ieeetr}

\end{document}

